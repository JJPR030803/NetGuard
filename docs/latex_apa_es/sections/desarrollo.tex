\subsection{Entorno de Desarrollo}
La Suite de Seguridad de Red se desarrolla utilizando un flujo de trabajo de desarrollo moderno con un enfoque en la calidad del código, pruebas y colaboración.

\subsubsection{Herramientas de Desarrollo}
Las siguientes herramientas se utilizan en el proceso de desarrollo:

\begin{itemize}
    \item \textbf{Poetry}: Gestión de dependencias y empaquetado
    \item \textbf{Git}: Control de versiones
    \item \textbf{GitHub}: Alojamiento de código y colaboración
    \item \textbf{Pre-commit}: Hooks de Git para verificaciones de calidad de código
    \item \textbf{Docker}: Contenedorización para desarrollo y pruebas
    \item \textbf{VS Code / PyCharm}: IDEs recomendados
\end{itemize}

\subsubsection{Configuración del Entorno de Desarrollo}
Para configurar un entorno de desarrollo:

\begin{lstlisting}[language=bash, caption=Configuración del Entorno de Desarrollo]
# Clonar el repositorio
git clone https://github.com/yourusername/network-security-suite.git
cd network-security-suite

# Instalar dependencias
poetry install

# Instalar hooks de pre-commit
poetry run pre-commit install

# Activar el entorno virtual
poetry shell
\end{lstlisting}

\subsection{Estructura del Código}
El código sigue una estructura modular para promover la mantenibilidad y la capacidad de prueba:

\begin{lstlisting}[language=bash, caption=Estructura del Proyecto]
network-security-suite/
├── src/
│   └── network_security_suite/
│       ├── __init__.py
│       ├── api/                 # Endpoints de API
│       │   ├── __init__.py
│       │   └── main.py
│       ├── core/                # Funcionalidad principal
│       │   └── __init__.py
│       ├── ml/                  # Modelos de aprendizaje automático
│       │   └── __init__.py
│       ├── models/              # Modelos de datos
│       │   └── __init__.py
│       ├── sniffer/             # Captura de paquetes
│       │   ├── __init__.py
│       │   └── packet_capture.py
│       ├── utils/               # Funciones de utilidad
│       │   └── __init__.py
│       ├── config.py            # Manejo de configuración
│       └── main.py              # Punto de entrada de la aplicación
├── tests/                       # Suites de pruebas
│   ├── __init__.py
│   ├── conftest.py
│   ├── e2e/                     # Pruebas de extremo a extremo
│   │   └── __init__.py
│   ├── integration/             # Pruebas de integración
│   │   └── __init__.py
│   └── unit/                    # Pruebas unitarias
│       └── __init__.py
├── docs/                        # Documentación
├── scripts/                     # Scripts de utilidad
├── .github/                     # Flujos de trabajo de GitHub
├── .pre-commit-config.yaml      # Configuración de pre-commit
├── pyproject.toml               # Metadatos del proyecto y dependencias
├── poetry.lock                  # Dependencias bloqueadas
├── Dockerfile                   # Configuración de Docker
├── docker-compose.yml           # Configuración de Docker Compose
└── README.md                    # Visión general del proyecto
\end{lstlisting}

\subsection{Estándares de Codificación}
El proyecto sigue estrictos estándares de codificación para garantizar la calidad y consistencia del código:

\subsubsection{Formato del Código}
El formato del código se aplica utilizando Black e isort:

\begin{lstlisting}[language=bash, caption=Formato del Código]
# Formatear código con Black
poetry run black .

# Ordenar importaciones con isort
poetry run isort .
\end{lstlisting}

\subsubsection{Linting}
La calidad del código se verifica utilizando Pylint y Flake8:

\begin{lstlisting}[language=bash, caption=Linting]
# Ejecutar Pylint
poetry run pylint src/

# Ejecutar Flake8
poetry run flake8 src/
\end{lstlisting}

\subsubsection{Verificación de Tipos}
La verificación estática de tipos se realiza utilizando MyPy:

\begin{lstlisting}[language=bash, caption=Verificación de Tipos]
# Ejecutar MyPy
poetry run mypy src/
\end{lstlisting}

\subsubsection{Escaneo de Seguridad}
Las vulnerabilidades de seguridad se verifican utilizando Bandit y Safety:

\begin{lstlisting}[language=bash, caption=Escaneo de Seguridad]
# Ejecutar Bandit
poetry run bandit -r src/

# Verificar dependencias para vulnerabilidades
poetry run safety check
\end{lstlisting}

\subsection{Pruebas}
La Suite de Seguridad de Red tiene una estrategia de pruebas integral que incluye pruebas unitarias, pruebas de integración y pruebas de extremo a extremo.

\subsubsection{Estructura de Pruebas}
Las pruebas se organizan en tres categorías:

\begin{itemize}
    \item \textbf{Pruebas Unitarias}: Prueban funciones y clases individuales de forma aislada
    \item \textbf{Pruebas de Integración}: Prueban interacciones entre componentes
    \item \textbf{Pruebas de Extremo a Extremo}: Prueban todo el sistema desde la perspectiva del usuario
\end{itemize}

\subsubsection{Ejecución de Pruebas}
Las pruebas se pueden ejecutar utilizando pytest:

\begin{lstlisting}[language=bash, caption=Ejecución de Pruebas]
# Ejecutar todas las pruebas
poetry run pytest

# Ejecutar solo pruebas unitarias
poetry run pytest tests/unit/

# Ejecutar solo pruebas de integración
poetry run pytest tests/integration/

# Ejecutar solo pruebas de extremo a extremo
poetry run pytest tests/e2e/

# Ejecutar pruebas con informe de cobertura
poetry run pytest --cov=src --cov-report=html
\end{lstlisting}

\subsubsection{Fixtures de Prueba}
Los fixtures comunes de prueba se definen en \texttt{tests/conftest.py}:

\begin{lstlisting}[language=python, caption=Ejemplo de Fixtures de Prueba]
import pytest
from fastapi.testclient import TestClient
from sqlalchemy import create_engine
from sqlalchemy.orm import sessionmaker
from sqlalchemy.pool import StaticPool

from network_security_suite.main import app
from network_security_suite.models.base import Base

@pytest.fixture
def client():
    """
    Crear un cliente de prueba para la aplicación FastAPI.
    """
    return TestClient(app)

@pytest.fixture
def db_session():
    """
    Crear una sesión de base de datos en memoria para pruebas.
    """
    engine = create_engine(
        "sqlite:///:memory:",
        connect_args={"check_same_thread": False},
        poolclass=StaticPool,
    )
    TestingSessionLocal = sessionmaker(autocommit=False, autoflush=False, bind=engine)
    Base.metadata.create_all(bind=engine)
    
    db = TestingSessionLocal()
    try:
        yield db
    finally:
        db.close()
\end{lstlisting}

\subsubsection{Escritura de Pruebas}
Las pruebas se escriben utilizando pytest y siguen un patrón consistente:

\begin{lstlisting}[language=python, caption=Ejemplo de Prueba]
import pytest
from network_security_suite.sniffer.packet_capture import PacketCapture

def test_packet_capture_initialization():
    """Probar que PacketCapture se inicializa correctamente."""
    capture = PacketCapture(interface="eth0")
    assert capture.interface == "eth0"
    assert not capture.is_running

def test_packet_capture_start_stop():
    """Probar iniciar y detener la captura de paquetes."""
    capture = PacketCapture(interface="eth0")
    
    # Simular la captura real de paquetes para evitar acceso a la red durante las pruebas
    with patch("network_security_suite.sniffer.packet_capture.sniff") as mock_sniff:
        capture.start()
        assert capture.is_running
        mock_sniff.assert_called_once()
        
        capture.stop()
        assert not capture.is_running
\end{lstlisting}

\subsection{Integración Continua}
El proyecto utiliza GitHub Actions para integración continua:

\begin{lstlisting}[language=yaml, caption=Ejemplo de Flujo de Trabajo de GitHub]
name: CI

on:
  push:
    branches: [ main ]
  pull_request:
    branches: [ main ]

jobs:
  test:
    runs-on: ubuntu-latest
    strategy:
      matrix:
        python-version: [3.9, 3.10, 3.11]

    steps:
    - uses: actions/checkout@v3
    - name: Set up Python ${{ matrix.python-version }}
      uses: actions/setup-python@v4
      with:
        python-version: ${{ matrix.python-version }}
    
    - name: Install Poetry
      run: |
        curl -sSL https://install.python-poetry.org | python3 -
    
    - name: Install dependencies
      run: |
        poetry install
    
    - name: Lint with flake8
      run: |
        poetry run flake8 src/
    
    - name: Type check with mypy
      run: |
        poetry run mypy src/
    
    - name: Security check with bandit
      run: |
        poetry run bandit -r src/
    
    - name: Test with pytest
      run: |
        poetry run pytest --cov=src --cov-report=xml
    
    - name: Upload coverage to Codecov
      uses: codecov/codecov-action@v3
      with:
        file: ./coverage.xml
\end{lstlisting}

\subsection{Proceso de Lanzamiento}
La Suite de Seguridad de Red sigue un proceso de lanzamiento estructurado:

\subsubsection{Versionado}
El proyecto utiliza Versionado Semántico (SemVer):

\begin{itemize}
    \item \textbf{Versión mayor}: Cambios incompatibles en la API
    \item \textbf{Versión menor}: Nueva funcionalidad de manera compatible con versiones anteriores
    \item \textbf{Versión de parche}: Correcciones de errores compatibles con versiones anteriores
\end{itemize}

\subsubsection{Pasos de Lanzamiento}
El proceso de lanzamiento involucra los siguientes pasos:

\begin{enumerate}
    \item Actualizar versión en \texttt{pyproject.toml}
    \item Actualizar CHANGELOG.md con notas de lanzamiento
    \item Crear una rama de lanzamiento (\texttt{release/vX.Y.Z})
    \item Ejecutar pruebas y verificaciones finales
    \item Fusionar la rama de lanzamiento a main
    \item Etiquetar el lanzamiento (\texttt{vX.Y.Z})
    \item Construir y publicar artefactos
    \item Actualizar documentación
\end{enumerate}

\begin{lstlisting}[language=bash, caption=Proceso de Lanzamiento]
# Actualizar versión en pyproject.toml
poetry version minor  # o major, patch

# Crear rama de lanzamiento
git checkout -b release/v$(poetry version -s)

# Confirmar cambios
git add pyproject.toml CHANGELOG.md
git commit -m "Release v$(poetry version -s)"

# Enviar rama
git push origin release/v$(poetry version -s)

# Después de la revisión de PR y fusión a main
git checkout main
git pull

# Etiquetar el lanzamiento
git tag -a v$(poetry version -s) -m "Release v$(poetry version -s)"
git push origin v$(poetry version -s)
\end{lstlisting}

\subsection{Documentación}
La documentación es una parte integral del proceso de desarrollo:

\subsubsection{Documentación de Código}
El código se documenta utilizando docstrings siguiendo el estilo de Google:

\begin{lstlisting}[language=python, caption=Ejemplo de Docstring]
def process_packet(packet, filter_criteria=None):
    """
    Procesa un paquete de red y extrae información relevante.
    
    Args:
        packet: El paquete a procesar (objeto de paquete Scapy).
        filter_criteria: Criterios opcionales para filtrar paquetes.
            Si se proporciona, solo se procesarán los paquetes que coincidan con los criterios.
            
    Returns:
        dict: Un diccionario que contiene información extraída del paquete.
        
    Raises:
        ValueError: Si el paquete está malformado o no se puede procesar.
    
    Example:
        >>> from scapy.all import IP, TCP, Ether
        >>> packet = Ether()/IP(src="192.168.1.1", dst="192.168.1.2")/TCP()
        >>> info = process_packet(packet)
        >>> print(info["src_ip"])
        192.168.1.1
    """
    # Implementación...
\end{lstlisting}

\subsubsection{Documentación de API}
Los endpoints de API se documentan utilizando las características de documentación incorporadas de FastAPI:

\begin{lstlisting}[language=python, caption=Ejemplo de Documentación de API]
@router.get("/packets/{packet_id}", response_model=PacketDetail)
async def get_packet(
    packet_id: str,
    current_user: User = Depends(get_current_user)
):
    """
    Recupera información detallada sobre un paquete específico.
    
    Parámetros:
    - **packet_id**: El identificador único del paquete
    
    Retorna:
    - **PacketDetail**: Información detallada del paquete
    
    Excepciones:
    - **404**: Si no se encuentra el paquete
    - **403**: Si el usuario no tiene permiso para ver el paquete
    """
    # Implementación...
\end{lstlisting}

\subsubsection{Documentación del Proyecto}
La documentación del proyecto se mantiene en el directorio \texttt{docs/} e incluye:

\begin{itemize}
    \item Guías de usuario
    \item Referencia de API
    \item Documentación de arquitectura
    \item Guías de desarrollo
    \item Guías de implementación
\end{itemize}

\subsection{Contribución}
Las contribuciones a la Suite de Seguridad de Red son bienvenidas. El proceso de contribución está documentado en \texttt{CONTRIBUTING.md} e incluye:

\begin{enumerate}
    \item Bifurcar el repositorio
    \item Crear una rama de características
    \item Realizar cambios
    \item Ejecutar pruebas y verificaciones
    \item Enviar una solicitud de extracción
    \item Abordar comentarios de revisión
\end{enumerate}

\begin{lstlisting}[language=bash, caption=Flujo de Trabajo de Contribución]
# Bifurcar el repositorio en GitHub

# Clonar tu bifurcación
git clone https://github.com/yourusername/network-security-suite.git
cd network-security-suite

# Crear una rama de características
git checkout -b feature/your-feature-name

# Realizar cambios y confirmar
git add .
git commit -m "Agregar descripción de tu característica"

# Enviar cambios a tu bifurcación
git push origin feature/your-feature-name

# Crear una solicitud de extracción en GitHub
\end{lstlisting}

Todas las contribuciones deben adherirse a los estándares de codificación del proyecto, pasar todas las pruebas e incluir documentación apropiada.