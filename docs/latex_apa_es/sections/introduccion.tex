\subsection{Visión General}
La Suite de Seguridad de Red es una solución de seguridad de red a nivel empresarial diseñada para proporcionar capacidades completas de monitoreo, análisis y detección de amenazas para entornos de red modernos. Al combinar el análisis de paquetes en tiempo real con algoritmos avanzados de aprendizaje automático, el sistema ofrece medidas de seguridad proactivas para identificar y mitigar posibles amenazas antes de que puedan causar daños significativos.

\subsection{Propósito}
El propósito principal de este sistema es mejorar la seguridad de la red a través de:
\begin{itemize}
    \item Monitoreo en tiempo real del tráfico de red
    \item Inspección y análisis profundo de paquetes
    \item Detección automatizada de amenazas utilizando aprendizaje automático
    \item Registro y reportes completos
    \item Visualización fácil de usar a través de un panel de control basado en React
\end{itemize}

\subsection{Características Principales}
La Suite de Seguridad de Red ofrece las siguientes características principales:
\begin{itemize}
    \item \textbf{Análisis de paquetes de red en tiempo real} utilizando Scapy para inspección profunda de paquetes
    \item \textbf{Detección de amenazas basada en aprendizaje automático} para identificar patrones anómalos y posibles amenazas de seguridad
    \item \textbf{API REST FastAPI} para integración con otros sistemas y servicios
    \item \textbf{Panel de control basado en React} para visualización y gestión intuitiva
    \item \textbf{Contenedorización Docker} para fácil despliegue y escalabilidad
    \item \textbf{Suite de pruebas completa} para garantizar confiabilidad y rendimiento
\end{itemize}

\subsection{Audiencia Objetivo}
Este sistema está diseñado para:
\begin{itemize}
    \item Administradores de red
    \item Equipos de operaciones de seguridad
    \item Profesionales de seguridad de TI
    \item Organizaciones que requieren monitoreo avanzado de seguridad de red
\end{itemize}

\subsection{Estructura del Documento}
Esta documentación está organizada para proporcionar una comprensión completa de la Suite de Seguridad de Red:
\begin{itemize}
    \item La Sección 2 describe la arquitectura general del sistema
    \item La Sección 3 detalla los componentes individuales y sus funciones
    \item Las Secciones 4 y 5 cubren la instalación, configuración y configuración
    \item La Sección 6 proporciona instrucciones de uso
    \item La Sección 7 documenta la referencia de la API
    \item La Sección 8 explica los modelos de aprendizaje automático utilizados
    \item Las Secciones 9-12 cubren desarrollo, seguridad, rendimiento y trabajo futuro
\end{itemize}