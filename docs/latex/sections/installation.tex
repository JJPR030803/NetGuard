\subsection{System Requirements}
Before installing the Network Security Suite, ensure your system meets the following requirements:

\subsubsection{Hardware Requirements}
\begin{itemize}
    \item \textbf{CPU}: Multi-core processor (4+ cores recommended for production)
    \item \textbf{RAM}: Minimum 8GB (16GB+ recommended for production)
    \item \textbf{Storage}: Minimum 20GB free space (SSD recommended)
    \item \textbf{Network}: Gigabit Ethernet interface
\end{itemize}

\subsubsection{Software Requirements}
\begin{itemize}
    \item \textbf{Operating System}: Linux (Ubuntu 20.04+, CentOS 8+), macOS 11+, or Windows 10/11 with WSL2
    \item \textbf{Python}: Version 3.9 or higher
    \item \textbf{Docker}: Version 20.10 or higher (for containerized deployment)
    \item \textbf{Docker Compose}: Version 2.0 or higher (for containerized deployment)
    \item \textbf{Poetry}: Version 1.2 or higher (for development)
    \item \textbf{Node.js}: Version 16 or higher (for frontend development)
    \item \textbf{npm}: Version 8 or higher (for frontend development)
\end{itemize}

\subsection{Installation Methods}
The Network Security Suite can be installed using one of the following methods:

\subsubsection{Method 1: Using Poetry (Recommended for Development)}
Poetry is the recommended tool for managing dependencies and virtual environments during development.

\begin{lstlisting}[language=bash, caption=Installation using Poetry]
# Clone the repository
git clone https://github.com/yourusername/network-security-suite.git
cd network-security-suite

# Install dependencies using Poetry
poetry install

# Activate the virtual environment
poetry shell
\end{lstlisting}

\subsubsection{Method 2: Using Docker (Recommended for Production)}
Docker provides an isolated environment with all dependencies pre-configured, making it ideal for production deployments.

\begin{lstlisting}[language=bash, caption=Installation using Docker]
# Clone the repository
git clone https://github.com/yourusername/network-security-suite.git
cd network-security-suite

# Build and start the containers
docker-compose up --build
\end{lstlisting}

\subsubsection{Method 3: Manual Installation}
For systems where Poetry or Docker cannot be used, manual installation is possible.

\begin{lstlisting}[language=bash, caption=Manual Installation]
# Clone the repository
git clone https://github.com/yourusername/network-security-suite.git
cd network-security-suite

# Create and activate a virtual environment
python -m venv venv
source venv/bin/activate  # On Windows: venv\Scripts\activate

# Install dependencies
pip install -r requirements.txt
\end{lstlisting}

\subsection{Post-Installation Setup}
After installing the Network Security Suite, complete the following setup steps:

\subsubsection{Database Setup}
The system requires a database for storing packet metadata and analysis results.

\begin{lstlisting}[language=bash, caption=Database Setup]
# Run database migrations
poetry run alembic upgrade head

# Or with Docker:
docker-compose exec app alembic upgrade head
\end{lstlisting}

\subsubsection{Initial Configuration}
Create an initial configuration file by copying the example configuration:

\begin{lstlisting}[language=bash, caption=Initial Configuration]
# Copy example configuration
cp config.example.yaml config.yaml

# Edit the configuration file
nano config.yaml  # Or use any text editor
\end{lstlisting}

\subsubsection{Network Interface Configuration}
Configure the network interfaces to be monitored:

\begin{lstlisting}[language=bash, caption=Network Interface Configuration]
# List available network interfaces
poetry run python -m network_security_suite.utils.list_interfaces

# Update the network interfaces in the configuration file
nano config.yaml  # Or use any text editor
\end{lstlisting}

\subsection{Verification}
Verify that the installation was successful by running the following commands:

\begin{lstlisting}[language=bash, caption=Installation Verification]
# Run the development server
poetry run uvicorn src.network_security_suite.main:app --reload

# Or with Docker:
docker-compose up

# Access the API documentation
# Open a web browser and navigate to: http://localhost:8000/docs
\end{lstlisting}

\subsection{Troubleshooting}
If you encounter issues during installation, try the following troubleshooting steps:

\begin{itemize}
    \item \textbf{Dependency Issues}: Ensure you have the correct versions of Python, Poetry, Docker, etc.
    \item \textbf{Permission Issues}: Ensure you have the necessary permissions to capture network packets (usually requires root/admin privileges).
    \item \textbf{Network Interface Issues}: Verify that the configured network interfaces exist and are accessible.
    \item \textbf{Port Conflicts}: Ensure that the ports used by the application (default: 8000) are not in use by other applications.
    \item \textbf{Log Files}: Check the log files in the \texttt{logs/} directory for error messages.
\end{itemize}

For more detailed troubleshooting information, refer to the troubleshooting guide in the project wiki.