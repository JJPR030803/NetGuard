\subsection{Starting the System}
The Network Security Suite can be started using different methods depending on your installation:

\subsubsection{Using Poetry}
\begin{lstlisting}[language=bash, caption=Starting with Poetry]
# Activate the virtual environment
poetry shell

# Start the API server
poetry run uvicorn src.network_security_suite.main:app --reload

# Start the packet sniffer (in a separate terminal)
poetry run python -m network_security_suite.sniffer.packet_capture
\end{lstlisting}

\subsubsection{Using Docker}
\begin{lstlisting}[language=bash, caption=Starting with Docker]
# Start all services
docker-compose up

# Or start in detached mode
docker-compose up -d
\end{lstlisting}

\subsubsection{Using Systemd (Linux)}
If you've installed the system as a service on Linux, you can use systemd:

\begin{lstlisting}[language=bash, caption=Starting with Systemd]
# Start the API service
sudo systemctl start network-security-api

# Start the packet sniffer service
sudo systemctl start network-security-sniffer

# Check status
sudo systemctl status network-security-api
sudo systemctl status network-security-sniffer
\end{lstlisting}

\subsection{Accessing the Dashboard}
The Network Security Suite provides a web-based dashboard for monitoring and management:

\begin{enumerate}
    \item Open a web browser
    \item Navigate to \texttt{http://localhost:3000} (or the configured dashboard URL)
    \item Log in using your credentials
\end{enumerate}

\begin{figure}[H]
    \centering

    \caption{Network Security Suite Dashboard}
    \label{fig:dashboard}
\end{figure}

\subsection{Dashboard Features}
The dashboard provides the following features:

\subsubsection{Network Traffic Overview}
The main dashboard page displays an overview of network traffic, including:

\begin{itemize}
    \item Real-time traffic volume graph
    \item Protocol distribution chart
    \item Top source and destination IP addresses
    \item Recent security alerts
\end{itemize}

\subsubsection{Packet Analysis}
The packet analysis page allows you to:

\begin{itemize}
    \item View detailed packet information
    \item Filter packets by various criteria (IP, port, protocol, etc.)
    \item Export packet data for further analysis
    \item Drill down into specific connections
\end{itemize}

\subsubsection{Threat Detection}
The threat detection page shows:

\begin{itemize}
    \item Detected security threats
    \item Anomaly detection results
    \item Historical threat trends
    \item Threat details and recommended actions
\end{itemize}

\subsubsection{System Configuration}
The configuration page allows you to:

\begin{itemize}
    \item Modify system settings
    \item Configure network interfaces
    \item Manage alerting rules
    \item Update machine learning parameters
\end{itemize}

\subsubsection{User Management}
The user management page allows administrators to:

\begin{itemize}
    \item Create and manage user accounts
    \item Assign roles and permissions
    \item Configure authentication settings
    \item View user activity logs
\end{itemize}

\subsection{Command Line Interface}
The Network Security Suite also provides a command-line interface (CLI) for various operations:

\begin{lstlisting}[language=bash, caption=CLI Examples]
# Show help
poetry run python -m network_security_suite --help

# Start packet capture
poetry run python -m network_security_suite capture --interface eth0

# Analyze a PCAP file
poetry run python -m network_security_suite analyze --file capture.pcap

# Generate a report
poetry run python -m network_security_suite report --output report.pdf

# Train ML models
poetry run python -m network_security_suite train --data training_data/
\end{lstlisting}

\subsection{API Usage}
The Network Security Suite provides a RESTful API that can be used for integration with other systems:

\subsubsection{Authentication}
To use the API, you first need to authenticate and obtain an access token:

\begin{lstlisting}[language=bash, caption=API Authentication]
# Obtain an access token
curl -X POST http://localhost:8000/api/auth/token \
  -H "Content-Type: application/x-www-form-urlencoded" \
  -d "username=admin&password=your_password"

# Response will contain the access token
# {
#   "access_token": "eyJhbGciOiJIUzI1NiIsInR5cCI6IkpXVCJ9...",
#   "token_type": "bearer",
#   "expires_in": 1800
# }
\end{lstlisting}

\subsubsection{API Endpoints}
Once authenticated, you can use the API endpoints:

\begin{lstlisting}[language=bash, caption=API Endpoint Examples]
# Get recent packets
curl -X GET http://localhost:8000/api/packets \
  -H "Authorization: Bearer YOUR_ACCESS_TOKEN"

# Get traffic statistics
curl -X GET http://localhost:8000/api/stats/traffic \
  -H "Authorization: Bearer YOUR_ACCESS_TOKEN"

# Get detected threats
curl -X GET http://localhost:8000/api/threats \
  -H "Authorization: Bearer YOUR_ACCESS_TOKEN"

# Configure network interface
curl -X PUT http://localhost:8000/api/config/network \
  -H "Authorization: Bearer YOUR_ACCESS_TOKEN" \
  -H "Content-Type: application/json" \
  -d '{"interfaces": ["eth0"], "promiscuous_mode": true}'
\end{lstlisting}

\subsection{Scheduled Tasks}
The Network Security Suite includes several scheduled tasks that run automatically:

\begin{itemize}
    \item \textbf{Database Maintenance}: Runs daily to optimize database performance
    \item \textbf{Model Retraining}: Runs weekly to update machine learning models
    \item \textbf{Report Generation}: Runs daily to generate summary reports
    \item \textbf{Log Rotation}: Runs daily to manage log files
\end{itemize}

These tasks can be configured in the \texttt{config.yaml} file under the \texttt{scheduler} section.

\subsection{Backup and Restore}
It's important to regularly back up your Network Security Suite data:

\begin{lstlisting}[language=bash, caption=Backup and Restore]
# Create a backup
poetry run python -m network_security_suite.utils.backup \
  --output backup_$(date +%Y%m%d).zip

# Restore from backup
poetry run python -m network_security_suite.utils.restore \
  --input backup_20230101.zip
\end{lstlisting}

\subsection{Troubleshooting}
If you encounter issues while using the Network Security Suite, try the following:

\begin{itemize}
    \item \textbf{Check Logs}: Examine the log files in the \texttt{logs/} directory
    \item \textbf{Verify Configuration}: Ensure your configuration is correct
    \item \textbf{Check System Resources}: Ensure your system has sufficient resources
    \item \textbf{Restart Services}: Try restarting the services
    \item \textbf{Update Dependencies}: Ensure all dependencies are up to date
\end{itemize}

For more detailed troubleshooting information, refer to the troubleshooting guide in the project wiki.