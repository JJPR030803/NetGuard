\subsection{High-Level Architecture}
The Network Security Suite follows a modular, microservices-based architecture designed for scalability, maintainability, and extensibility. The system is composed of several key components that work together to provide comprehensive network security monitoring and threat detection.

\begin{figure}[H]
    \centering

    \caption{High-level architecture of the Network Security Suite}
    \label{fig:architecture}
\end{figure}

\subsection{Core Components}
The architecture consists of the following core components:

\begin{itemize}
    \item \textbf{Packet Sniffer Module}: Captures and processes network packets in real-time using Scapy
    \item \textbf{Analysis Engine}: Performs deep packet inspection and preliminary analysis
    \item \textbf{Machine Learning Module}: Applies ML algorithms to detect anomalies and potential threats
    \item \textbf{API Layer}: Provides RESTful endpoints for integration and data access
    \item \textbf{Database}: Stores packet metadata, analysis results, and system configuration
    \item \textbf{Frontend Dashboard}: Provides visualization and management interface
\end{itemize}

\subsection{Data Flow}
The data flows through the system as follows:

\begin{enumerate}
    \item Network packets are captured by the Packet Sniffer Module
    \item Captured packets are processed and relevant metadata is extracted
    \item Packet metadata is stored in the database and forwarded to the Analysis Engine
    \item The Analysis Engine performs initial analysis based on predefined rules
    \item The Machine Learning Module analyzes patterns to detect anomalies
    \item Analysis results are stored in the database
    \item The API Layer provides access to the data for the Frontend Dashboard and external systems
    \item The Frontend Dashboard visualizes the data and alerts for user interaction
\end{enumerate}

\subsection{Deployment Architecture}
The Network Security Suite is designed to be deployed in various configurations:

\begin{itemize}
    \item \textbf{Standalone Deployment}: All components run on a single machine
    \item \textbf{Distributed Deployment}: Components are distributed across multiple machines for improved performance and scalability
    \item \textbf{Containerized Deployment}: Components run in Docker containers, managed by Docker Compose or Kubernetes
\end{itemize}

\subsection{Technology Stack}
The system is built using the following technologies:

\begin{itemize}
    \item \textbf{Backend}: Python 3.9+
    \item \textbf{Packet Capture}: Scapy 2.5.0+
    \item \textbf{API Framework}: FastAPI 0.104.1+
    \item \textbf{ASGI Server}: Uvicorn with standard extras
    \item \textbf{Data Validation}: Pydantic 2.5.0+
    \item \textbf{Data Processing}: Pandas 2.1.0+, NumPy 1.24.0+
    \item \textbf{Machine Learning}: Scikit-learn 1.3.0+
    \item \textbf{Asynchronous Processing}: Asyncio 3.4.3+
    \item \textbf{Authentication}: Python-jose with cryptography, Passlib with bcrypt
    \item \textbf{Database ORM}: SQLAlchemy 2.0.0+
    \item \textbf{Database Migrations}: Alembic 1.12.0+
    \item \textbf{Frontend}: React (JavaScript/TypeScript)
    \item \textbf{Containerization}: Docker, Docker Compose
\end{itemize}

\subsection{Security Architecture}
The security architecture of the system includes:

\begin{itemize}
    \item \textbf{Authentication}: JWT-based authentication for API access
    \item \textbf{Authorization}: Role-based access control for different user types
    \item \textbf{Encryption}: TLS/SSL for all communications
    \item \textbf{Secure Storage}: Encrypted storage for sensitive data
    \item \textbf{Audit Logging}: Comprehensive logging of all system activities
\end{itemize}