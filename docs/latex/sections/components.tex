\subsection{Packet Sniffer Module}
The Packet Sniffer Module is responsible for capturing and processing network packets in real-time. It is implemented in the \texttt{network\_security\_suite.sniffer} package.

\subsubsection{Key Features}
\begin{itemize}
    \item Real-time packet capture using Scapy
    \item Support for multiple network interfaces
    \item Packet filtering based on configurable rules
    \item Packet metadata extraction
    \item Efficient packet processing pipeline
\end{itemize}

\subsubsection{Implementation Details}
The module uses Scapy's packet capture capabilities to intercept network traffic. It implements a multi-threaded architecture to ensure high-performance packet processing without dropping packets during high traffic periods.

\begin{lstlisting}[language=Python, caption=Example Packet Capture Code]
from scapy.all import sniff, IP, TCP

def packet_callback(packet):
    if IP in packet and TCP in packet:
        # Process packet
        src_ip = packet[IP].src
        dst_ip = packet[IP].dst
        src_port = packet[TCP].sport
        dst_port = packet[TCP].dport
        # Store or forward packet metadata
        
# Start packet capture
sniff(prn=packet_callback, filter="tcp", store=0)
\end{lstlisting}

\subsection{Analysis Engine}
The Analysis Engine performs deep packet inspection and preliminary analysis based on predefined rules. It is implemented in the \texttt{network\_security\_suite.core} package.

\subsubsection{Key Features}
\begin{itemize}
    \item Rule-based packet analysis
    \item Protocol-specific inspection
    \item Traffic pattern recognition
    \item Signature-based threat detection
    \item Real-time alerting for suspicious activities
\end{itemize}

\subsubsection{Implementation Details}
The Analysis Engine uses a combination of rule-based analysis and pattern matching to identify potential security threats. It supports custom rule definitions and can be extended with additional analysis capabilities.

\subsection{Machine Learning Module}
The Machine Learning Module applies ML algorithms to detect anomalies and potential threats that may not be detected by traditional rule-based approaches. It is implemented in the \texttt{network\_security\_suite.ml} package.

\subsubsection{Key Features}
\begin{itemize}
    \item Anomaly detection using unsupervised learning
    \item Classification of known attack patterns
    \item Behavioral analysis of network traffic
    \item Continuous learning from new data
    \item Model versioning and management
\end{itemize}

\subsubsection{Implementation Details}
The module uses scikit-learn for implementing various machine learning algorithms. It includes preprocessing pipelines, feature extraction, model training, and prediction components.

\subsection{API Layer}
The API Layer provides RESTful endpoints for integration and data access. It is implemented using FastAPI in the \texttt{network\_security\_suite.api} package.

\subsubsection{Key Features}
\begin{itemize}
    \item RESTful API endpoints
    \item Authentication and authorization
    \item Rate limiting and request validation
    \item Comprehensive API documentation using Swagger/OpenAPI
    \item Asynchronous request handling
\end{itemize}

\subsubsection{Implementation Details}
The API is built using FastAPI, which provides automatic validation, serialization, and documentation. It follows RESTful principles and uses JWT for authentication.

\begin{lstlisting}[language=Python, caption=Example API Endpoint]
from fastapi import APIRouter, Depends, HTTPException
from typing import List

router = APIRouter()

@router.get("/packets", response_model=List[PacketSchema])
async def get_packets(
    limit: int = 100,
    current_user: User = Depends(get_current_user)
):
    """
    Retrieve recent packet data.
    """
    if not current_user.has_permission("read:packets"):
        raise HTTPException(status_code=403, detail="Not authorized")
    
    packets = await get_recent_packets(limit)
    return packets
\end{lstlisting}

\subsection{Database}
The database stores packet metadata, analysis results, and system configuration. The system uses SQLAlchemy as an ORM to interact with the database.

\subsubsection{Key Features}
\begin{itemize}
    \item Efficient storage of packet metadata
    \item Indexing for fast query performance
    \item Support for multiple database backends
    \item Schema migrations using Alembic
    \item Connection pooling for optimal performance
\end{itemize}

\subsubsection{Implementation Details}
The database schema is defined using SQLAlchemy models in the \texttt{network\_security\_suite.models} package. Alembic is used for managing database migrations.

\subsection{Frontend Dashboard}
The Frontend Dashboard provides visualization and management interface for the system. It is implemented as a React application.

\subsubsection{Key Features}
\begin{itemize}
    \item Real-time data visualization
    \item Interactive network traffic analysis
    \item Alert management and notification
    \item User and permission management
    \item System configuration interface
    \item Responsive design for different device sizes
\end{itemize}

\subsubsection{Implementation Details}
The dashboard is built using React with modern JavaScript/TypeScript. It communicates with the backend API to retrieve and display data, and to manage system configuration.